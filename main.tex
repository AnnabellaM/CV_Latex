\documentclass[10pt, letterpaper]{article}

% Packages:
\usepackage[
    ignoreheadfoot, % set margins without considering header and footer
    top=2 cm, % seperation between body and page edge from the top
    bottom=2 cm, % seperation between body and page edge from the bottom
    left=2 cm, % seperation between body and page edge from the left
    right=2 cm, % seperation between body and page edge from the right
    footskip=1.0 cm, % seperation between body and footer
    % showframe % for debugging 
]{geometry} % for adjusting page geometry
\usepackage{titlesec} % for customizing section titles
\usepackage{tabularx} % for making tables with fixed width columns
\usepackage{array} % tabularx requires this
\usepackage[dvipsnames]{xcolor} % for coloring text
\definecolor{primaryColor}{RGB}{0, 79, 144} % define primary color
\usepackage{enumitem} % for customizing lists
\usepackage{fontawesome5} % for using icons
\usepackage{amsmath} % for math
\usepackage[
    pdftitle={Miao Miao's CV},
    pdfauthor={Miao Miao},
    pdfcreator={LaTeX with RenderCV},
    colorlinks=true,
    urlcolor=primaryColor
]{hyperref} % for links, metadata and bookmarks
\usepackage[pscoord]{eso-pic} % for floating text on the page
\usepackage{calc} % for calculating lengths
\usepackage{bookmark} % for bookmarks
\usepackage{lastpage} % for getting the total number of pages
\usepackage{changepage} % for one column entries (adjustwidth environment)
\usepackage{paracol} % for two and three column entries
\usepackage{ifthen} % for conditional statements
\usepackage{needspace} % for avoiding page brake right after the section title
\usepackage{iftex} % check if engine is pdflatex, xetex or luatex

% Ensure that generate pdf is machine readable/ATS parsable:
\ifPDFTeX
    \input{glyphtounicode}
    \pdfgentounicode=1
    % \usepackage[T1]{fontenc} % this breaks sb2nov
    \usepackage[utf8]{inputenc}
    \usepackage{lmodern}
\fi



% Some settings:
\AtBeginEnvironment{adjustwidth}{\partopsep0pt} % remove space before adjustwidth environment
\pagestyle{empty} % no header or footer
\setcounter{secnumdepth}{0} % no section numbering
\setlength{\parindent}{0pt} % no indentation
\setlength{\topskip}{0pt} % no top skip
\setlength{\columnsep}{0cm} % set column seperation
\makeatletter
\let\ps@customFooterStyle\ps@plain % Copy the plain style to customFooterStyle
\patchcmd{\ps@customFooterStyle}{\thepage}{
    \color{gray}\textit{\small Miao Miao - Page \thepage{} of \pageref*{LastPage}}
}{}{} % replace number by desired string
\makeatother
\pagestyle{customFooterStyle}

\titleformat{\section}{\needspace{4\baselineskip}\bfseries\large}{}{0pt}{}[\vspace{1pt}\titlerule]

\titlespacing{\section}{
    % left space:
    -1pt
}{
    % top space:
    0.3 cm
}{
    % bottom space:
    0.2 cm
} % section title spacing

\renewcommand\labelitemi{$\circ$} % custom bullet points
\newenvironment{highlights}{
    \begin{itemize}[
        topsep=0.10 cm,
        parsep=0.10 cm,
        partopsep=0pt,
        itemsep=0pt,
        leftmargin=0.4 cm + 10pt
    ]
}{
    \end{itemize}
} % new environment for highlights

\newenvironment{highlightsforbulletentries}{
    \begin{itemize}[
        topsep=0.10 cm,
        parsep=0.10 cm,
        partopsep=0pt,
        itemsep=0pt,
        leftmargin=10pt
    ]
}{
    \end{itemize}
} % new environment for highlights for bullet entries


\newenvironment{onecolentry}{
    \begin{adjustwidth}{
        0.2 cm + 0.00001 cm
    }{
        0.2 cm + 0.00001 cm
    }
}{
    \end{adjustwidth}
} % new environment for one column entries

\newenvironment{twocolentry}[2][]{
    \onecolentry
    \def\secondColumn{#2}
    \setcolumnwidth{\fill, 4.5 cm}
    \begin{paracol}{2}
}{
    \switchcolumn \raggedleft \secondColumn
    \end{paracol}
    \endonecolentry
} % new environment for two column entries

\newenvironment{header}{
    \setlength{\topsep}{0pt}\par\kern\topsep\centering\linespread{1.5}
}{
    \par\kern\topsep
} % new environment for the header

\newcommand{\placelastupdatedtext}{% \placetextbox{<horizontal pos>}{<vertical pos>}{<stuff>}
  \AddToShipoutPictureFG*{% Add <stuff> to current page foreground
    \put(
        \LenToUnit{\paperwidth-2 cm-0.2 cm+0.05cm},
        \LenToUnit{\paperheight-1.0 cm}
    ){\vtop{{\null}\makebox[0pt][c]{
        \small\color{gray}\textit{Last updated in March 2025}\hspace{\widthof{Last updated in September 2024}}
    }}}%
  }%
}%

% save the original href command in a new command:
\let\hrefWithoutArrow\href

% new command for external links:
\renewcommand{\href}[2]{\hrefWithoutArrow{#1}{\ifthenelse{\equal{#2}{}}{ }{#2 }\raisebox{.15ex}{\footnotesize \faExternalLink*}}}


\begin{document}
    \newcommand{\AND}{\unskip
        \cleaders\copy\ANDbox\hskip\wd\ANDbox
        \ignorespaces
    }
    \newsavebox\ANDbox
    \sbox\ANDbox{}

    \placelastupdatedtext
    \begin{header}
        \textbf{\fontsize{24 pt}{24 pt}\selectfont Miao Miao}

        \vspace{0.3 cm}

        \normalsize
        \mbox{{\color{black}\footnotesize\faMapMarker*}\hspace*{0.13cm}Dallas, TX}%
        \kern 0.25 cm%
        \AND%
        \kern 0.25 cm%
        \mbox{\hrefWithoutArrow{mailto:youremail@yourdomain.com}{\color{black}{\footnotesize\faEnvelope[regular]}\hspace*{0.13cm}mmiao@utdallas.edu}}%
        \kern 0.25 cm%
        \AND%
        \kern 0.25 cm%
        % \mbox{\hrefWithoutArrow{tel:+90-541-999-99-99}{\color{black}{\footnotesize\faPhone*}\hspace*{0.13cm}469-922-7794}}%
        % \kern 0.25 cm%
        % \AND%
        % \kern 0.25 cm%
        \mbox{\hrefWithoutArrow{https://annabellam.github.io/}{\color{black}{\footnotesize\faLink}\hspace*{0.13cm}annabellam.github.io}}%
        \kern 0.25 cm%
        \AND%
        \kern 0.25 cm%
        \mbox{\hrefWithoutArrow{https://linkedin.com/in/annabellamiao}{\color{black}{\footnotesize\faLinkedinIn}\hspace*{0.13cm}annabellamiao}}%
        \kern 0.25 cm%
        \AND%
        \kern 0.25 cm%
        \mbox{\hrefWithoutArrow{https://github.com/AnnabellaM}{\color{black}{\footnotesize\faGithub}\hspace*{0.13cm}AnnabellaM}}%
    \end{header}

    \vspace{0.3 cm - 0.3 cm}


    \section{RESEARCH INTERESTS}
        
        \begin{onecolentry}
        My research interest lies in program analysis and fuzz testing, and their applications in software reliability and security. I focus on enhancing the reliability and usability of static analysis tools by automatically detecting bugs and diagnosing their root causes. I also work on improving the fuzz testing evaluation process by developing benchmarks that integrate program characteristics, aiming for a more comprehensive and accurate assessment of fuzzing tools.
        \end{onecolentry}

    \section{EDUCATION}
        
        \begin{twocolentry}{
            
            
        \textit{Jan 2023 - present}}
            \textbf{Doctor of Philosophy, Software Engineering }

            \textit{The University of Texas at Dallas, Richardson, Texas, USA}
            
            \textit{Advisor: Dr. Shiyi Wei}
        \end{twocolentry}

        \vspace{0.20 cm}
        
        \begin{twocolentry}{
            
            
        \textit{Aug 2021 - Dec 2022}
        
        GPA: 3.97
        
        }
        
            \textbf{Master of Science, Software Engineering }

            \textit{The University of Texas at Dallas, Richardson, Texas, USA}
        \end{twocolentry}

        \vspace{0.20 cm}
        
        \begin{twocolentry}{
            
            
        \textit{Sep 2014 - June 2018}
        
        GPA: 3.59 
        
        }
            \textbf{Bachelor of Engineering, Computer Science and Technology}

            \textit{The Xi’an University of Finance and Economics, Xi’an, China}
        \end{twocolentry}


        \section{AWARDS}
    
        \begin{onecolentry}
            \begin{highlights}
                \item The Jonsson School Best Teaching Assistant Award in 2024.
                \item The ACM SIGSOFT CAPS Travel Award for ICSE 2025.
            \end{highlights}
        \end{onecolentry}


        
    \section{PUBLICATIONS}

    \begin{onecolentry}
        \textit{* ICSE and ISSTA are top-tier conferences in Software Engineering, while TOSEM and EMSE are among the field's leading journals.}
    \end{onecolentry}

    \vspace{0.20 cm}

        % \begin{samepage}
            \begin{onecolentry}
                \textbf{• Program Feature-based Benchmarking for Fuzz Testing}

                % \vspace{0.10 cm}

                \mbox{\textbf{Miao Miao}}, \mbox{Sriteja Kummita}, \mbox{Eric Bodden}, and \mbox{Shiyi Wei}.
                
                % \vspace{0.10 cm}
                
                \textit{In the 34th ACM SIGSOFT International Symposium on Software Testing and Analysis (ISSTA), 2025.}
            \end{onecolentry}


            \vspace{0.20 cm}
            
            \begin{onecolentry}
                \textbf{• An Extensive Empirical Study of Nondeterministic Behavior in Static Analysis Tools}

                % \vspace{0.10 cm}

                \mbox{\textbf{Miao Miao}}, \mbox{Austin Mordahl}, \mbox{Dakota Soles}, \mbox{Alice Beideck}, and \mbox{Shiyi Wei}.
                
                % \vspace{0.10 cm}
                
                \textit{In the 47th IEEE/ACM International Conference on Software Engineering (ICSE), 2025.}
            \end{onecolentry}

            \vspace{0.20 cm}
            
            \begin{onecolentry}
                \textbf{• Program Feature-based Fuzzing Benchmarking}

                % \vspace{0.10 cm}

                \mbox{\textbf{Miao Miao}}.
                
                % \vspace{0.10 cm}
                
                \textit{In the 47th IEEE/ACM International Conference on Software Engineering, ACM Student Research Competition (ICSE-SRC), 2025.}
            \end{onecolentry}

            \vspace{0.20 cm}
            
            \begin{onecolentry}
                \textbf{• Visualization Task Taxonomy to Understand the Fuzzing Internals (Registered Report)}

                % \vspace{0.10 cm}

                \mbox{Kummita Sriteja}, \mbox{\textbf{Miao Miao}}, \mbox{Bodden Eric}, and \mbox{Shiyi Wei}.
                
                % \vspace{0.10 cm}
                
                \textit{In the Proceedings of the 3rd ACM International Fuzzing Workshop (FUZZING), 2024.}
            \end{onecolentry}

            \vspace{0.20 cm}
            
            \begin{onecolentry}
                \textbf{• ECSTATIC: Automatic Configuration-Aware Testing and Debugging of Static Analysis Tools}

                % \vspace{0.10 cm}

                \mbox{Austin Mordahl}, \mbox{Dakota Soles}, \mbox{\textbf{Miao Miao}}, \mbox{Zenong Zhang}, and \mbox{Shiyi Wei}.
                
                % \vspace{0.10 cm}
                
                \textit{In the 47th IEEE/ACM International Conference on Software Engineering (ICSE), 2025.}
            \end{onecolentry}

            \vspace{0.20 cm}

            \begin{onecolentry}
                \textbf{\underline{Papers Under Review}}

            \end{onecolentry}

            \vspace{0.20 cm}
            
            \begin{onecolentry}
                \textbf{• Visualization Task Taxonomy to Understand the Fuzzing Internals}

                % \vspace{0.10 cm}

                \mbox{Kummita Sriteja}, \mbox{\textbf{Miao Miao}}, \mbox{Bodden Eric}, and \mbox{Shiyi Wei}.
                
                % \vspace{0.10 cm}
                
                \textit{Under review at ACM Transactions on Software Engineering and Methodology Journal, 2025.}
            \end{onecolentry}

            \vspace{0.20 cm}
            
            \begin{onecolentry}
                \textbf{• Towards Automated Identification of Data Constraints in Software Documentation}

                % \vspace{0.10 cm}

                \mbox{Ying Zhou}, \mbox{\textbf{Miao Miao}}, \mbox{Vlad Birsan}, \mbox{Oscar Chaparro}, \mbox{Shiyi Wei}, and \mbox{Andrian Marcus}.
                
                % \vspace{0.10 cm}
                
                \textit{Under review at Empirical Software Engineering Journal, 2025.}
            \end{onecolentry}

        % \end{samepage}

        
    \section{RESEARCH EXPERIENCE}
        
        \begin{twocolentry}{
        \textit{Jan 2025 – Present}}
            \textbf{Fuzzing Bottleneck Localization}
        \end{twocolentry}

        \vspace{0.10 cm}
        \begin{onecolentry}
            \begin{highlights}
                \item Conduct experiments to identify fuzzing blockers across different fuzzers (e.g., AFL++, LibFuzzer, Honggfuzz) using Fuzz-Introspector.  
                \item Analyze the impact of various factors on fuzzing blocker detection, including branch side hit frequency, number of trials, and runtime.  
                \item Investigate and localize the root causes of fuzzing blockers specific to each fuzzer.  
                \item Perform differential testing to evaluate how fuzzer design influences fuzzing blockers.  
            \end{highlights}
        \end{onecolentry}

        \vspace{0.2 cm}


        \begin{twocolentry}{
        \textit{May 2024 – Nov 2024}}
            \textbf{Program Feature-based Benchmarking for Fuzz Testing}
        \end{twocolentry}

        \vspace{0.10 cm}
        \begin{onecolentry}
            \begin{highlights}
                \item Performed a literature review of 25 recent grey-box fuzzing papers to extract fine-grained program features from their claimed improvements. 
                \item Created the first feature-based benchmark that defines 10 configurable parameters for the extracted program features with 153 generated programs. 
                \item Evaluated 11 popular fuzzers to understand fuzzer behaviors and the impact of each program parameter on their performance. 

            \end{highlights}
        \end{onecolentry}

        \vspace{0.2 cm}

        \begin{twocolentry}{
        \textit{May 2024 – Nov 2024}}
            \textbf{Visualization Task Taxonomy to Understand the Fuzzing Internals}
        \end{twocolentry}

        \vspace{0.10 cm}
        \begin{onecolentry}
            \begin{highlights}
            \item Conducted semi-structured interviews with fuzzing experts. 
            \item Systematically extracted the task taxonomy from the interview data through qualitative data analysis. 
            \item Evaluated the support of existing visualization tools for fuzzing through the lens of our taxonomy. 
            \end{highlights}
        \end{onecolentry}


        \vspace{0.2 cm}

        \begin{twocolentry}{
        \textit{June 2023 – Aug 2024}}
            \textbf{An Extensive Empirical Study of Nondeterministic Behavior in Static Analysis Tools}
        \end{twocolentry}

        \vspace{0.10 cm}
        \begin{onecolentry}
            \begin{highlights}

            \item Performed qualitative analysis of the repositories of 11 popular static analysis tools that shows common nondeterministic issues and categorizes their root causes.
            \item Constructed an experiment framework and conducted empirical study that detects previously unknown nondeterministic behaviors in tools such as SOOT, WALA, DOOP, FlowDroid, PyCG and Infer.
            \item Debugged root causes of discovered nondeterministic bugs and reported them to tool developers.
            \end{highlights}
        \end{onecolentry}

        \vspace{0.2 cm}

        \begin{twocolentry}{
        \textit{Sep 2022 – Dec 2023}}
            \textbf{Towards Automated Identification of Data Constraints in Software Documentation}
        \end{twocolentry}

        \vspace{0.10 cm}
        \begin{onecolentry}
            \begin{highlights}
            \item Identified and validated the data constraints in requirements documentation which specify allowed data values in software systems.
            \item Identified 15 discourse patterns, commonly used to describe data constraints in natural language. 
            \item Debugged and evaluated an NLP-based automated splitter which breaks down a sentence to fragments based on the developed discourse patterns. These patterns are used as features for machine learning.
            \item Trained and evaluated 5 machine learning classifiers for automatically extracting data constraints.
            \end{highlights}
        \end{onecolentry}


    \section{INDUSTRY EXPERIENCE}

        \begin{twocolentry}{
        \textit{May 2022 – Aug 2022}
        }
        
        \textbf{iOS Engineering Intern, }Tinder Inc.  
        
        \end{twocolentry}

        \vspace{0.2 cm}

        \begin{twocolentry}{
        \textit{Aug 2019 – June 2021}
        }
        
        \textbf{iOS Developer}, LotusFlare Inc.   
        
        \end{twocolentry}

        \vspace{0.2 cm}

        \begin{twocolentry}{
        \textit{Apr 2018 – Apr 2019}
        }
        
        \textbf{Junior Software Engineer}, KA Software 
        
        \end{twocolentry}

        \vspace{0.2 cm}


    \section{TEACHING EXPERIENCE}
    
    \textbf{Teaching Assistant}
    
    \textit{* Contributed to curriculum development and design of projects and assignments; Provided video and in-class tutorials to guide students through project implementation.}
    
    \begin{onecolentry}
        \begin{highlights}
        \item CS/SE 6356: Software Maintenance Evolution and Re-Engineering (Spring 2023)
        \item CS 4386: Compiler Design (Fall 2023)
        \item CS 6353: Compiler Construction (Spring 2024)
        \end{highlights}
    \end{onecolentry}

    \section{SERVICES}
    \begin{onecolentry}
        \begin{highlights}
        \item Artifact Evaluation Committee member: the 46th ACM SIGPLAN Conference on Programming Language Design and Implementation (PLDI 2025).

        \item Junior Program Committee member: the 22nd International Conference on Mining Software Repositories (MSR 2025).
        \item Mentored four high school students and three undergraduate students in the UTD K-12 Summer Research Program and Clarks Summer Research Program (Summer 2023, Summer 2024).
        \item Student Volunteer: the 47th IEEE/ACM International Conference on Software Engineering (ICSE 2025).
        \item Student Volunteer: the 32nd ACM SIGSOFT International Symposium on Software Testing and Analysis (ISSTA 2023).
        \item Sub-review: CCS2025, USENIX2025, ISSTA2025, ICSE2025, FSE2025, USENIX2024, MSR2025, ISSTA2024, ICSE2024, FSE2024, ASE2024, FSE 2023, ISSTA 2023, SecDev 2023, ICSE 2023.
        \end{highlights}
    \end{onecolentry}
    
\end{document}
